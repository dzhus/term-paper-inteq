\documentclass{article}
\usepackage[utf8x]{inputenc}
\usepackage[english,russian]{babel}

\usepackage{amsmath,amssymb}

% rich title
\usepackage{titling}

\numberwithin{equation}{section}

% Russian traditions
\renewcommand{\epsilon}{\varepsilon}
\renewcommand{\phi}{\varphi}
\renewcommand{\leq}{\leqslant}
\renewcommand{\geq}{\geqslant}
\usepackage{misccorr}

\providecommand{\at}[1]{\vert_{#1}}
\begin{document}

\author{Дмитрий Джус}
\title{Курсовая работа по теме \\
  \Huge{«Интегральные уравнения»}}
\pretitle{\begin{center}\LARGE}
\posttitle{\par\end{center}\vskip 3pc}
\maketitle
\thispagestyle{empty}

\clearpage
\tableofcontents

\clearpage
\part{Постановка задачи}

\section{Исходные данные}

В настоящей курсовой работе рассматриваются методы решения
неоднородного симметричного интегрального уравнения вида

\begin{equation}
  \label{eq:ieqgen}
  \phi(x) - \lambda \int \limits_a^b {K(x, t) \phi(t) dt} = f(x)
\end{equation}

В решении использованы конкретные значения $a=0$, $b=1$, $\lambda = 2$
и $f(x) = \ch(x)$, при которых \eqref{eq:ieqgen} принимает вид

\begin{equation}
  \label{eq:ieq}
  \phi(x) - 2 \int \limits_0^1 {K(x, t) \phi(t) dt} = \ch{x}
\end{equation}

Где ядро $K(x,t)$ определено следующим образом:

\begin{equation}
  \label{eq:kernel}
  K(x, t) = 
  \begin{cases}
    \frac{\ch{x} \ch(t-1)}{\sh(1)} & 0 \leq x \leq t\\
    \frac{\ch{t} \ch(x-1)}{\sh(1)} & t < x \leq 1
  \end{cases}
\end{equation}

\section{Используемые методы}

Точное аналитическое решение было получено путём сведения уравнения с
неоднородной краевой задаче. Также предложено приближённое решение в
виде ряда по собственным функциям оператора $\hat{I}(x)\at{s} =
\int_a^b{K(s, \tau) x(\tau) d\tau}$ и численное решение, полученное
заменой определённого интеграла приближённой квадратурной формулой
трапеций.

Сравнение полученных приближённых решений с точным представлено в
разделе \pageref{sec:comparison}.

\clearpage
\part{Решение}

\section{Аналитическое решение}

\section{Решение в виде ряда по собственным функциям интегрального
  оператора}

\section{Численное решение}

\section{Сопоставление результатов}
\label{sec:comparison}

\end{document}
