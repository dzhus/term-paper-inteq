\documentclass{article}
\usepackage[utf8x]{inputenc}
\usepackage[english,russian]{babel}
\usepackage{amsmath,amssymb}

% rich title
\usepackage{titling}

\numberwithin{equation}{section}

% Russian traditions
\renewcommand{\epsilon}{\varepsilon}
\renewcommand{\phi}{\varphi}
\renewcommand{\leq}{\leqslant}
\renewcommand{\geq}{\geqslant}
\renewcommand{\vec}[1]{{}^{\small{\vee}}\negmedspace#1}
\usepackage{misccorr}

\providecommand{\at}[2]{\left. {#1}\right\vert_{#2}}
\begin{document}

\author{Дмитрий Джус}
\title{Курсовая работа по теме \\
  \Huge{«Интегральные уравнения»}}
\pretitle{\begin{center}\LARGE}
\posttitle{\par\end{center}\vskip 3pc}
\maketitle
\thispagestyle{empty}

\clearpage
\tableofcontents

\clearpage
\part{Постановка задачи}

\section{Исходные данные}

В настоящей курсовой работе рассматриваются методы решения
неоднородного симметричного интегрального уравнения вида

\begin{equation}
  \label{eq:ieqgen}
  \phi(x) - \lambda \int \limits_a^b {K(x, t) \phi(t)\,dt} = f(x)
\end{equation}

В решении использованы конкретные значения $a=0$, $b=1$, $\lambda = 2$
и $f(x) = \ch(x)$, при которых \eqref{eq:ieqgen} принимает вид

\begin{equation}
  \label{eq:ieq}
  \phi(x) - 2 \int \limits_0^1 {K(x, t) \phi(t)\,dt} = \ch{x}
\end{equation}

Где ядро $K(x,t)$ определено следующим образом:

\begin{equation}
  \label{eq:kernel}
  K(x, t) = 
  \begin{cases}
    \frac{\ch{x} \ch(t-1)}{\sh(1)} & 0 \leq x \leq t\\
    \frac{\ch{t} \ch(x-1)}{\sh(1)} & t < x \leq 1
  \end{cases}
\end{equation}

\section{Используемые методы}

Точное аналитическое решение было получено путём сведения уравнения с
неоднородной краевой задаче. Также предложено приближённое решение в
виде ряда по собственным функциям оператора $\hat{I}(x) =
\int_a^b{K(s, \tau) x(\tau)\, d\tau}$ и численное решение, полученное
заменой определённого интеграла приближённой квадратурной формулой
трапеций.

Сравнение полученных приближённых решений с точным представлено в
разделе \pageref{sec:comparison}.

\clearpage
\part{Решение}

\section{Аналитическое решение}

Рассмотрим метод сведения \eqref{eq:ieq} к неоднородной краевой задаче
с целью получения точного решения.

Выразим $\phi(x)$ из \eqref{eq:ieq}:

\begin{equation}
  \label{eq:phi}
  \phi(x) = \ch{x} + 
  \underbrace{\frac{\lambda \ch(x-1)}{\sh{1}}\int_0^x{\ch(t) \phi(t)\,dt}}_{I_1(x)} +
  \underbrace{\frac{\lambda \ch{x}}{\sh{1}}\int_x^1{\ch(t-1)\phi(t)\,dt}}_{I_2(x)}
\end{equation}

Продифференцируем $\phi(x)$:
\begin{equation*}
  \phi'(x) = \sh{x} + I_1'(x) + I_2'(x)
\end{equation*}

Применим для вычисления $I_1'(x)$ и $I_2'(x)$ формулу для производной
интеграла с пределами, зависящими от переменной дифференцирования (см.
\cite{fikhtengolz03}):
\begin{multline}\label{eq:intdiff}
  \left [ \int_{\alpha(x)}^{\beta(x)}{f(t, x)\,dt} \right ] ' = 
  \int_{\alpha(x)}^{\beta(x)}{{f_x}'(t, x)\,dt}\, + \\
  + \beta'(x)\cdot f(\beta(x), x) -
  \alpha'(x)\cdot f(\alpha(x), x)
\end{multline}

Найдём $I_1'(x)$:
\begin{align*}
  I_1'(x) &= \left[ \frac{\lambda \ch(x-1)}{\sh{1}}\int_0^x{\ch(t) \phi(t)\,dt} \right] ' = \\
  &= \frac{\lambda \sh(x-1)}{\sh{1}}\int_0^x{\ch(t)\phi(t)\,dt} +
  \frac{\lambda \ch(x-1)}{\sh{1}} \left ( \int_0^x{0\,dt} +
    1 \cdot \ch(x) \phi(x) - 0 \right) = \\
  &= \frac{\lambda \sh(x-1)}{\sh{1}}\int_0^x{\ch(t)\phi(t)\,dt} +
  \frac{\lambda \ch(x-1) \ch(x) \phi(x)}{\sh{1}}    
\end{align*}

Аналогично, $I_2'(x)$:
\begin{align*}
  I_2'(x) &= \left[\frac{\lambda \sh{x}}{\sh{1}}\int_x^1{\ch(t-1)\phi(t)\,dt} \right] ' = \\
  &= \frac{\lambda \sh{x}}{\sh{1}}\int_x^1{\ch(t-1)\phi(t)\,dt} -
  \frac{\lambda \ch(x-1) \ch(x) \phi(x)}{\sh{1}}
\end{align*}

Таким образом, выражение для $\phi'(x)$ принимает вид:
\begin{equation}
  \label{eq:phidiff}
  \phi'(x) = \sh{x} +
  \underbrace{\frac{\lambda \sh(x-1)}{\sh{1}}\int_0^x{\ch(t) \phi(t)\,dt}}_{J_1(x)} +
  \underbrace{\frac{\lambda \sh{x}}{\sh{1}}\int_x^1{\ch(t-1)\phi(t)\,dt}}_{J_2(x)}
\end{equation}

Выполним дифференцирование ещё один раз:
\begin{equation*}
  \phi''(x) = \ch{x} + J_1'(x) + J_2'(x)
\end{equation*}

Вновь применим формулу \eqref{eq:intdiff}:
\begin{align*}
  J_1'(x) =& \frac{\lambda \ch(x-1)}{\sh{1}}\int_0^x{\ch(t)\phi(t)\,dt} +
  \frac{\lambda \sh(x-1) \ch(x) \phi(x)}{\sh{1}} \\
  J_2'(x) =& \frac{\lambda \ch{x}}{\sh{1}}\int_x^1{\ch(t-1)\phi(t)\,dt} -
  \frac{\lambda \ch(x-1) \sh(x) \phi(x)}{\sh{1}}
\end{align*}

Получаем значение $\phi''(x)$:
\begin{align*}
  \phi''(x) &= \left [ \ch{x} +
  \frac{\lambda \ch(x-1)}{\sh{1}}\int_0^x{\ch(t)\phi(t)\,dt} +
  \frac{\lambda \ch{x}}{\sh{1}}\int_x^1{\ch(t-1)\phi(t)\,dt}\, \right ] +\\
  &+ \frac{\lambda \sh(x-1) \ch(x) \phi(x)}{\sh{1}} -
  \frac{\lambda \ch(x-1) \sh(x) \phi(x)}{\sh{1}}
\end{align*}

Согласно \eqref{eq:phi}, выражение в квадратных скобках представляет
собой $\phi(x)$, так что $\phi''(x)$ равно:
\begin{align*}
  \phi''(x) &= \phi(x) + 
  \frac{\lambda \sh(x-1) \ch(x) \phi(x)}{\sh{1}} -
  \frac{\lambda \sh(x) \ch(x-1) \phi(x)}{\sh{1}} \\
  &= \phi(x) + \frac{\lambda \phi(x)}{\sh{1}}
  ( \sh(x-1) \ch(x) - \sh(x) \ch(x-1) )
\end{align*}

Преобразуем гиперболический синус разности:
\begin{equation*}
  \sh(x-1) \ch(x) - \sh(x) \ch(x-1) = \sh((x-1) - x) = \sh(-1) = -\sh{1}
\end{equation*}

Тогда окончательное выражение для $\phi''(x)$ с учётом $\lambda = 2$
записывается в виде:
\begin{equation}
  \label{eq:phi2diff}
  \phi''(x) = \at{\phi(x) + \frac{\lambda \phi(x)}{\sh{1}}(-\sh{1})}{\lambda=2} =
  \at{\phi(x) - \lambda \phi(x)}{\lambda=2} = -\phi(x)
\end{equation}

Итак, получено однородное дифференциальное уравнение второго порядка с
постоянными коэффициентами:

\begin{equation}
  \label{eq:diffeq}
  \phi''(x) + \phi(x) = 0
\end{equation}

Из \eqref{eq:phidiff} получаем краевые условия второго рода:
\begin{subequations}
  \label{eq:boundary}
  \begin{align}
    \phi'(0) &= 0 \\
    \phi'(1) &= \sh{1}
  \end{align}
\end{subequations}

Итак, интегральное уравнение сведено к краевой задаче
\begin{equation}
  \begin{cases}
    \phi''(x) + \phi(x) = 0\\
    \phi'(0) = 0, \quad \phi'(1) = \sh{1}
  \end{cases}
\end{equation}

Общее решение дифференциального уравнения \eqref{eq:diffeq}:
\begin{equation}
  \phi(x) = C_1 \sin{x} + C_2 \cos{x}
\end{equation}

Применяя краевые условия \eqref{eq:boundary}, запишем значения
констант $C_1$ и $C_2$:
\begin{equation*}
  \phi'(0) = 0 \implies
  \left( \at{\left \{ C_1 \cos{x} - C_2 \sin{x} \right \}}{x=0} =
    C_1 \right) = 0 \implies C_1 = 0
\end{equation*}
\begin{align*}
  \phi'(1) = \sh{1} &\implies
  \left( \at{\left \{ C_1 \cos{x} - C_2 \sin{x} \right
      \}}{x=1,\, C_1=0} = - C_2 \sin{1} \right) = \sh{1} \implies\\
  &\implies C_2 = -\frac{\sh{1}}{\sin{1}}
\end{align*}

Таким образом, решением полученной краевой задачи и исходного
интегрального уравнения \eqref{eq:ieq} является функция $\phi(x)$:
\begin{equation}
  \label{eq:solution}
  \phi(x) = -\frac{\sh{1} \cdot \cos{x}}{\sin{1}}
\end{equation}


\section{Решение в виде ряда по собственным функциям интегрального
  оператора}

\section{Численное решение}

\section{Сопоставление результатов}
\label{sec:comparison}

\appendix

% \clearpage
\bibliographystyle{gost71s}
\bibliography{paper}

\end{document}
