\documentclass[11pt]{article}
\usepackage[utf8x]{inputenc}
\usepackage[english,russian]{babel}
\usepackage{amsmath,amssymb}
\usepackage[top=2.5cm, bottom=3cm, left=5cm, right=5cm]{geometry}

% literate programming tool
\usepackage{noweb}

% rich title
\usepackage{titling}

\numberwithin{equation}{section}

% Russian traditions
\renewcommand{\epsilon}{\varepsilon}
\renewcommand{\phi}{\varphi}
\renewcommand{\leq}{\leqslant}
\renewcommand{\geq}{\geqslant}
\newcommand{\intl}{\int\limits}
\usepackage{misccorr}

% TikZ
\usepackage{tikz}
\usepackage{pgfplots}

% Bib in TOC
\usepackage[numbib,nottoc]{tocbibind}

% Custom commands
\renewcommand{\vec}[1]{{}^{\vee}\negmedspace#1}
\providecommand{\pardiff}[2]{\frac{\partial{#1}}{\partial{#2}}}
\providecommand{\neword}{\emph}
\providecommand{\at}[2]{\left. {#1}\right\vert_{#2}}
\providecommand{\intat}[3]{\left. {#1}\right\vert^{#2}_{#3}}
\providecommand{\program}[1]{{\tt #1}}
\providecommand{\abs}[1]{\left \lvert{#1}\right \rvert}
\providecommand{\set}[1]{\mathbb{#1}}

\usepackage[unicode,
pdftex, colorlinks, linkcolor=blue, citecolor=teal,
pdfauthor=Dmitry Dzhus]{hyperref}

\begin{document}

\author{Дмитрий Джус}
\title{Курсовая работа по теме \\
  \Huge{«Интегральные уравнения»}}
\pretitle{\begin{center}\LARGE}
  \posttitle{\par\end{center}\vskip 3pc}
\date{}
\maketitle
\thispagestyle{empty}

\clearpage
\tableofcontents

\clearpage
\section{Предмет работы}
\label{sec:problem}
В настоящей курсовой работе рассматриваются методы решения
симметричного неоднородного интегрального уравнения Фредгольма второго
рода вида
\begin{equation}
  \label{eq:ieqgen}
  \phi(x) - \lambda \intl_a^b {K(x, t) \phi(t)\,dt} = f(x)
\end{equation}

Точное аналитическое решение было получено путём сведения уравнения с
неоднородной краевой задаче. Также предложено приближённое решение в
виде ряда по собственным функциям оператора $\hat{I}(f) =
\int_a^b{K(x, t) f(t)\, dt}$ и численное решение, полученное заменой
определённого интеграла приближённой квадратурной формулой трапеций.
Использованные методы рассмотрены в \cite{polyanin03}.

Полученные приближённые решения сопоставлены с точным.

В решении использованы конкретные значения $a=0$, $b=1$, $\lambda = 2$
и $f(x) = \ch(x)$, при которых \eqref{eq:ieqgen} принимает вид
\begin{equation}
  \label{eq:ieq}
  \phi(x) - 2 \intl_0^1 {K(x, t) \phi(t)\,dt} = \ch{x}
\end{equation}

Ядро $K(x,t)$ определено следующим образом:
\begin{equation}
  \label{eq:kernel}
  K(x, t) = 
  \begin{cases}
    \frac{\ch{x} \ch(t-1)}{\sh(1)} & 0 \leq x \leq t\\
    \frac{\ch{t} \ch(x-1)}{\sh(1)} & t < x \leq 1
  \end{cases}
\end{equation}

Рассматриваемое ядро является симметричным, поскольку для него
выполнено $K(x, t) = K(t, x)$, и фредгольмовым в силу непрерывности в
квадрате $S={0\leq x \leq 1,\, 0 \leq t \leq 1}$. Кроме того, раз
$K(x, t)$ представимо в виде суммы $g_1(x)h_1(t) + \dotsb + g_n(x)
h_n(t)$ (в данном случае $n=1$), ядро вырождено. Таким образом,
уравнение \eqref{eq:ieq} является интегральным уравнением Фредгольма
второго рода с симметричным вырожденным ядром.

\clearpage
\section{Аналитическое решение}
% >:-)
\label{sec:anal-solution}

Рассмотрим метод сведения \eqref{eq:ieq} к неоднородной краевой задаче
с целью получения точного решения.

Выразим $\phi(x)$ из \eqref{eq:ieq}:

\begin{equation}
  \label{eq:phi}
  \phi(x) = \ch{x} + 
  \underbrace{\frac{\lambda \ch(x-1)}{\sh{1}}\intl_0^x{\ch(t) \phi(t)\,dt}}_{I_1(x)} +
  \underbrace{\frac{\lambda \ch{x}}{\sh{1}}\intl_x^1{\ch(t-1)\phi(t)\,dt}}_{I_2(x)}
\end{equation}

Продифференцируем $\phi(x)$:
\begin{equation*}
  \phi'(x) = \sh{x} + I_1'(x) + I_2'(x)
\end{equation*}

Применим для вычисления $I_1'(x)$ и $I_2'(x)$ формулу для производной
интеграла с пределами, зависящими от переменной дифференцирования (см.
\cite{fikhtengolz03}):
\begin{multline}\label{eq:intdiff}
  \left [ \intl_{\alpha(x)}^{\beta(x)}{f(t, x)\,dt} \right ] ' = 
  \intl_{\alpha(x)}^{\beta(x)}{{f_x}'(t, x)\,dt}\, + \\
  + \beta'(x)\cdot f(\beta(x), x) -
  \alpha'(x)\cdot f(\alpha(x), x)
\end{multline}

Найдём $I_1'(x)$:
\begin{align*}
  I_1'(x) &= \left[ \frac{\lambda \ch(x-1)}{\sh{1}}\intl_0^x{\ch(t) \phi(t)\,dt} \right] ' = \\
  &= \frac{\lambda \sh(x-1)}{\sh{1}}\intl_0^x{\ch(t)\phi(t)\,dt} +
  \frac{\lambda \ch(x-1)}{\sh{1}} \left ( \intl_0^x{0\,dt} +
    1 \cdot \ch(x) \phi(x) - 0 \right) = \\
  &= \frac{\lambda \sh(x-1)}{\sh{1}}\intl_0^x{\ch(t)\phi(t)\,dt} +
  \frac{\lambda \ch(x-1) \ch(x) \phi(x)}{\sh{1}}    
\end{align*}

Аналогично, $I_2'(x)$:
\begin{align*}
  I_2'(x) &= \left[\frac{\lambda \sh{x}}{\sh{1}}\intl_x^1{\ch(t-1)\phi(t)\,dt} \right] ' = \\
  &= \frac{\lambda \sh{x}}{\sh{1}}\intl_x^1{\ch(t-1)\phi(t)\,dt} -
  \frac{\lambda \ch(x-1) \ch(x) \phi(x)}{\sh{1}}
\end{align*}

Таким образом, выражение для $\phi'(x)$ принимает вид:
\begin{equation}
  \label{eq:phidiff}
  \phi'(x) = \sh{x} +
  \underbrace{\frac{\lambda \sh(x-1)}{\sh{1}}\intl_0^x{\ch(t) \phi(t)\,dt}}_{J_1(x)} +
  \underbrace{\frac{\lambda \sh{x}}{\sh{1}}\intl_x^1{\ch(t-1)\phi(t)\,dt}}_{J_2(x)}
\end{equation}

Выполним дифференцирование ещё один раз:
\begin{equation*}
  \phi''(x) = \ch{x} + J_1'(x) + J_2'(x)
\end{equation*}

Вновь применим формулу \eqref{eq:intdiff}:
\begin{align*}
  J_1'(x) =& \frac{\lambda \ch(x-1)}{\sh{1}}\intl_0^x{\ch(t)\phi(t)\,dt} +
  \frac{\lambda \sh(x-1) \ch(x) \phi(x)}{\sh{1}} \\
  J_2'(x) =& \frac{\lambda \ch{x}}{\sh{1}}\intl_x^1{\ch(t-1)\phi(t)\,dt} -
  \frac{\lambda \ch(x-1) \sh(x) \phi(x)}{\sh{1}}
\end{align*}

Получаем значение $\phi''(x)$:
\begin{align*}
  \phi''(x) &= \left [ \ch{x} +
    \frac{\lambda \ch(x-1)}{\sh{1}}\intl_0^x{\ch(t)\phi(t)\,dt} +
    \frac{\lambda \ch{x}}{\sh{1}}\intl_x^1{\ch(t-1)\phi(t)\,dt}\, \right ] +\\
  &+ \frac{\lambda \sh(x-1) \ch(x) \phi(x)}{\sh{1}} -
  \frac{\lambda \ch(x-1) \sh(x) \phi(x)}{\sh{1}}
\end{align*}

Согласно \eqref{eq:phi}, выражение в квадратных скобках представляет
собой $\phi(x)$, так что $\phi''(x)$ равно:
\begin{align*}
  \phi''(x) &= \phi(x) + 
  \frac{\lambda \sh(x-1) \ch(x) \phi(x)}{\sh{1}} -
  \frac{\lambda \sh(x) \ch(x-1) \phi(x)}{\sh{1}} \\
  &= \phi(x) + \frac{\lambda \phi(x)}{\sh{1}}
  ( \sh(x-1) \ch(x) - \sh(x) \ch(x-1) )
\end{align*}

Преобразуем гиперболический синус разности:
\begin{equation*}
  \sh(x-1) \ch(x) - \sh(x) \ch(x-1) = \sh((x-1) - x) = \sh(-1) = -\sh{1}
\end{equation*}

Тогда окончательное выражение для $\phi''(x)$ с учётом $\lambda = 2$
записывается в виде:
\begin{equation}
  \label{eq:phi2diff}
  \phi''(x) = \at{\phi(x) + \frac{\lambda \phi(x)}{\sh{1}}(-\sh{1})}{\lambda=2} =
  \at{\phi(x) - \lambda \phi(x)}{\lambda=2} = -\phi(x)
\end{equation}

Итак, получено однородное дифференциальное уравнение второго порядка с
постоянными коэффициентами:

\begin{equation}
  \label{eq:diffeq}
  \phi''(x) + \phi(x) = 0
\end{equation}

Из \eqref{eq:phidiff} получаем краевые условия второго рода:
\begin{subequations}
  \label{eq:boundary}
  \begin{align}
    \phi'(0) &= 0 \\
    \phi'(1) &= \sh{1}
  \end{align}
\end{subequations}

Итак, интегральное уравнение сведено к краевой задаче
\begin{equation}
  \begin{cases}
    \phi''(x) + \phi(x) = 0\\
    \phi'(0) = 0, \quad \phi'(1) = \sh{1}
  \end{cases}
\end{equation}

Общее решение дифференциального уравнения \eqref{eq:diffeq}:
\begin{equation}
  \phi(x) = C_1 \sin{x} + C_2 \cos{x}
\end{equation}

Применяя краевые условия \eqref{eq:boundary}, запишем значения
констант $C_1$ и $C_2$:
\begin{equation*}
  \phi'(0) = 0 \implies
  \left( \at{\left \{ C_1 \cos{x} - C_2 \sin{x} \right \}}{x=0} =
    C_1 \right) = 0 \implies C_1 = 0
\end{equation*}
\begin{align*}
  \phi'(1) = \sh{1} &\implies
  \left( \at{\left \{ C_1 \cos{x} - C_2 \sin{x} \right
      \}}{x=1,\, C_1=0} = - C_2 \sin{1} \right) = \sh{1} \implies\\
  &\implies C_2 = -\frac{\sh{1}}{\sin{1}}
\end{align*}

Таким образом, решением полученной краевой задачи и исходного
интегрального уравнения \eqref{eq:ieq} является функция $\phi(x)$:
\begin{equation}
  \label{eq:anal-solution}
  \phi(x) = -\frac{\sh{1} \cdot \cos{x}}{\sin{1}}
\end{equation}

\clearpage
\section{Решение в виде ряда по собственным\\
  функциям интегрального оператора}

\subsection{Теорема Гильберта—Шмидта}
\label{sec:gs}

Функция $\psi_i(x)$ называется \neword{собственной функцией}
интегрального оператора $\hat{I}(f) = \int_a^b{K(x, t) f(t)\, dt}$
(ядра $K(x, t)$), если она является нетривиальным решением однородного
интегрального уравнения
\begin{equation}
  \label{eq:eigfun}
  \psi_i(x) = \lambda_i \intl_a^b{K(x, t) \psi_i(t)\, dt}
\end{equation}
При этом $\lambda_i$ называется \neword{характеристическим числом},
соответствующим собственной функции $\psi_i(x)$, а обратная величина
$\frac{1}{\lambda_i}$ — \neword{собственным числом}.
Характеристические числа симметричного ядра действительны.
Последовательность собственных функций симметричного ядра
\emph{ортогональна} и её можно сделать ортонормированной.

Теорема Гильберта—Шмидта гласит (см. \cite{krasnov75}), что если
функция $f(x)$ представима в виде
\begin{equation}
  g(x) = \intl_a^b{K(x, t) \phi(t)\, dt}
\end{equation}
где симметричное ядро $K(x, t)$ и функция $g(t)$ квадратично
интегрируемы, то $f(x)$ можно разложить в ряд Фурье относительно
ортонормированной системы собственных функций ядра $K(x, t)$:
\begin{equation}
  \label{eq:gs-series}
  g(x) = \sum_{k=0}^{\infty}a_k\psi_k(x)
\end{equation}
где коэффициенты $a_k$ вычисляются следующим образом:
\begin{equation}
  a_k = \intl_a^b{g(x) \psi_k(x)\,dx} = \intl_a^b{\frac{\phi(x)}{\lambda_k}\psi_k(x)\,dx}
\end{equation}

Если при этом выполнено условие
\begin{equation}
  \intl_a^b{K^2(x, t)\,dt} \leq A < \infty
\end{equation}
то ряд \eqref{eq:gs-series} сходится абсолютно и равномерно для всякой
функции $g(x)$, удовлетворяющей условиям теоремы.

Сходимость ряда \eqref{eq:gs-series} равномерная и в случае
непрерывности ядра $K(x, t)$ и функции $g(x)$.

\subsection{Использование теоремы}
Рассмотрим способ решения интегрального уравнения с применением
теоремы Гильберта—Шмидта, изложенный в \cite{polyanin03}.

Пусть ядро уравнения \eqref{eq:ieqgen} представимо в виде равномерно
сходящегося ряда по ортонормированной системе своих собственных
функций:
\begin{equation}
  K(x, t) = \sum_{k=0}^{\infty}{\Bigl [ \smallint_a^b{K(x, t)\psi_k(t)\,
      dt} \Bigr ] \psi_k(t)}
\end{equation}

С учётом \eqref{eq:eigfun} этот ряд приведём к виду
\begin{equation}
  \label{eq:bilinear-kernel-series}
  K(x, t) = \sum_{k=0}^{\infty}{\frac{\psi_k(x)\psi_k(t)}{\lambda_k}}
\end{equation}

Преобразуем уравнение \eqref{eq:ieqgen} к виду
\begin{equation}
  \label{eq:gs-step1}
  \phi(x)-f(x) = \lambda \intl_a^b{K(x, t) \phi(t)\,dt}
\end{equation}

Применим к функции $\phi(x)-f(x)$ теорему Гильберта—Шмидта:
\begin{equation}
  \label{eq:gs-step2}
  \phi(x)-f(x) = \sum_{k=0}^{\infty}{a_k \psi_k(x)}
\end{equation}

Введём обозначения для $a_k = \int_a^b{[\phi(x)-f(x)]\psi_k(x)\,dx}$:
\begin{equation}
  \label{eq:gs-defs}
  a_k = \intl_a^b{\phi(x)\psi_k(x)\,dx} - \intl_a^b{f(x)\psi_k(x)\,dx}
  = \phi_k-f_k
\end{equation}

Теперь преобразуем правую часть \eqref{eq:gs-step1} с учётом
\eqref{eq:bilinear-kernel-series} и обозначений \eqref{eq:gs-defs}:
\begin{multline}
  \label{eq:gs-step3}
  \lambda\intl_a^b{K(x, t) \phi(t)\,dt} =
  \lambda \intl_a^b\sum_{k=0}^\infty{\frac{\psi_k(x)\psi_k(t)}{\lambda_k}\phi(t)\,dt} =\\
  = \lambda \sum_{k=0}^\infty \biggl [ \intl_a^b{\phi(t)\psi_k(t)\,dt}
  \biggr ] \frac{\psi_k(x)}{\lambda_k} = \lambda \sum_{k=0}^\infty
  \frac{\phi_k \psi_k(x)}{\lambda_k}
\end{multline}

Приравняв коэффициенты при $\psi_k(x)$ в \eqref{eq:gs-step2} и
\eqref{eq:gs-step3} и используя обозначение из \eqref{eq:gs-defs},
получим выражение для $a_k$:
\begin{equation*}
  a_k = \phi_k-f_k = \lambda\frac{\phi_k}{\lambda_k}
\end{equation*}
откуда следует
\begin{equation}
  a_k = \frac{\lambda f_k}{\lambda_k - \lambda}
\end{equation}

Значит, согласно \eqref{eq:gs-step1}
\begin{equation}
  \label{eq:gs-solution-form}
  \phi(x) = f(x) + \lambda\sum_{k=0}^\infty{\frac{f_k \psi_k(x)}{\lambda_k-\lambda}}
\end{equation}

Заметим, что для справедливости \eqref{eq:gs-solution-form} параметр
$\lambda$ в уравнении не должен быть характеристическим числом.
Случай, когда это условие не выполняется, подробно освещён в
\cite{krasnov76} и \cite{polyanin03}.

\subsection{Применение метода}
\label{sec:gs-application}

Известно (см. \cite{polyanin03}), что предположение о представимости
ядра $K(x, t)$ в виде ряда \eqref{eq:bilinear-kernel-series}
справедливо, если это ядро симметрично, непрерывно в квадрате
$S=\{0\leq x \leq 1,\, 0 \leq t \leq 1\}$ и имеет в $S$ равномерно
ограниченные частные производные.

Симметричность и непрерывность рассматриваемого ядра \eqref{eq:kernel}
очевидны. Кроме того, в квадрате $S$ частные производные ядра
равномерно ограничены, поскольку в нём
\begin{equation*}
  \abs{\pardiff{K}{x} = \frac{\sh(x)\ch(t-1)}{\sh 1}} \leq \ch1 \qquad
  \abs{\pardiff{K}{t} = \frac{\ch(x)\sh(t-1)}{\sh 1}} \leq \ch1
\end{equation*}

Кроме того, с учётом обозначенных свойств и квадратичной
интегрируемости ядра $K(x, t)$ применима и теорема Гильберта—Шмидта, а
в силу непрерывности функции $f(x)=\ch{x}$ существуют интегралы $f_k$
из \eqref{eq:gs-defs}. Так как $\phi(x)$ и $f(x)$ непрерывны, ряд
\eqref{eq:gs-step2} сходится равномерно согласно условиям,
обозначенным в разделе \ref{sec:gs}.

Таким образом, для построения решения по формуле
\eqref{eq:gs-solution-form} требуется:
\begin{enumerate}
\item Найти собственные функции $\psi_k(x)$ и характеристические
  значения $\lambda_k$ ядра $K(x, t)$
\item Проверить, что параметр уравнения $\lambda \notin {\lambda_k}$
\item Вычислить коэффициенты $f_k = \intl_a^b{f(x)\psi_k(x)\,dx}$
\end{enumerate}

\subsubsection{Построение ортонормированной системы\\
  собственных функций}

Для поиска собственных функций ядра \eqref{eq:kernel} сведём
интегральное уравнение \eqref{eq:eigfun} к краевой задаче так, как это
было сделано в разделе \ref{sec:anal-solution}, получив в результате
\begin{equation}
  \label{eq:eig-boundary}
  \begin{cases}
    \psi_k''(x)+(\lambda_k-1)\psi_k(x)=0 \\
    \psi_k'(0) = 0, \quad \psi_k'(1)=0
  \end{cases}
\end{equation}

Для решения данной краевой задачи рассмотрим три области значений
характеристических чисел $\lambda_k$.
\begin{enumerate}
\item $\lambda_k=1$

  Присвоим такое значение характеристическому числу $\lambda_0$.
  Уравнение \eqref{eq:eig-boundary} приводится к виду
  \begin{equation*}
    \psi_0''(x)=0
  \end{equation*}

  Его общее решение имеет вид
  \begin{equation*}
    \psi_0(x) = C_1x+C_2
  \end{equation*}

  Из краевых условий получаем $C_1=0 \implies \psi_0(x) = C_2$, в целях
  нормировки выберем значение $C_2 = 1$.

  Итак, характеристическому значению $\lambda_0=1$ соответствует
  собственная функция (очевидно нормированная)
  \begin{equation*}
    \psi_0(x) = 1
  \end{equation*}

\item $\lambda_k > 1$
  
  Обозначим $\lambda_k-1 = h^2 > 0$, получив уравнение
  \begin{equation*}
    \psi_k''(x)+h^2\psi_k(x) = 0
  \end{equation*}
  здесь $k = 1, 2, \dotsc$

  Его общим решением является функция вида
  \begin{equation*}
    \psi_k(x) = C_1 \cos{hx} + C_2 \sin{hx}
  \end{equation*}

  Из первого краевого условия получим
  \begin{equation*}
    \psi_k'(0) = 0 \implies \left( \at{\left \{ -h C_1 \sin{hx} + h C_2 \cos{hx} \right \}}{x=0} =
      h C_2 \right) = 0 \implies C_2 = 0
  \end{equation*}
  откуда $\psi_k(x) = C_1 \cos{hx}$. Для поиска $C_1$ воспользуемся
  вторым краевым условием
  \begin{equation*}
    \psi_k'(1) = 0 \implies \left( \at{\left \{ -h C_1 \sin{hx} \right \}}{x=1} =
      - h C_1 \sin{h} \right) = 0
  \end{equation*}
  откуда $h = \pi n, n \in \set{Z}$, поскольку иначе $C_1 = 0$ и
  решение получается тривиальным, что противоречит определению
  собственных функций.

  С учётом выполненной замены $h^2 = \lambda_k - 1$,
  \begin{equation*}
    \lambda_k = \pi^2 n^2 + 1,\, n\in\set{Z}
  \end{equation*}

  Заметим, что при значения $\lambda_k$ не меняются от смены знака $n$
  и при $n = 0$ не выполняется условие $\lambda_k > 1$. Поэтому можно
  взять лишь значения $n \in \set{N}$. Тогда не ограничивая общности
  можно положить $n = k$, получив следующую систему характеристических
  чисел:
  \begin{equation*}
    \lambda_k = \pi^2 k^2 + 1,\, k\in\set{N}
  \end{equation*}
  
  Соответствующая система собственных функций определяется следующим
  образом:
  \begin{equation*}
    \psi_k = C_1 \cos{\pi k x}
  \end{equation*}

  Нормировку ${\psi_k}$ выполним, выбрав параметр $C_1 = \sqrt{2}$,
  поскольку в таком случае
  \begin{equation*}
    {C_1}^2 \intl_0^1 {\cos^2{\pi k x}\, dx} = 2 \cdot \frac{1}{2} = 1
  \end{equation*}

  Таким образом, в данной области найдена ортонормированная система
  собственных функций и характеристических чисел
  \begin{align*}
    \psi_k = \sqrt{2} \cos{\pi k x} \\
    \lambda_k = \pi^2 k^2 + 1
  \end{align*}
  где $k = 1, 2, \dotsc$

\item $\lambda_k < 1$
  
  Вводя обозначение $1 - \lambda_k = h^2$, получим уравнение
  \begin{equation*}
    \psi_k''(x) - h^2 \psi_k(x) = 0
  \end{equation*}
  
  Для него известен общий вид решения:
  \begin{equation*}
    \psi_k(x) = C_1 \ch{h x} + C_2 \sh{h x}
  \end{equation*}

  Из первого краевого условия следует
  \begin{equation*}
    \psi_k'(0) = 0 \implies \left( \at{\left \{ h C_1 \sh{h x} + h
          C_2 \ch{h x} \right \}}{x=0} =
      h C_2 \right) = 0 \implies C_2 = 0
  \end{equation*}
  так что $\psi_k(x) = C_1 \ch{h x}$. Согласно второму краевому
  условию,
  \begin{equation*}
    \psi_k'(1) = 0 \implies \left( \at{\left \{ h C_1 \sh{hx} \right \}}{x=1} =
      h C_1 \sh{h} \right) = 0
  \end{equation*}
  
  Поскольку $C_1 ≠ 0$, получаем $h = 0 \implies \lambda_k = 1$, что
  противоречит предположению $\lambda_k < 1$. Значит, в данной области
  у рассматриваемого интегрального оператора характеристических
  значений нет.
\end{enumerate}

Итак, в результате решения краевой задачи \eqref{eq:eig-boundary}
найдена ортонормированная система собственных функций $\{\psi_i(x)\}$:
\begin{equation}
  \label{eq:eigfuns}
  \begin{cases}
    \psi_0(x) = 1 \\
    \psi_k(x) = \sqrt{2} \cos{\pi k x}, & k \in \set{N}
  \end{cases}
\end{equation}
с соответствующими характеристическими значениями
\begin{equation}
  \label{eq:eigvals}
  \begin{cases}
    \lambda_0 = 1 \\
    \lambda_k = \pi^2 k^2 + 1, & k \in \set{N}
  \end{cases}
\end{equation}

Из \eqref{eq:eigvals} следует, что данное по условию значение
параметра $\lambda = 2$ не принадлежит $\{\lambda_i\}$, так что
формула \eqref{eq:gs-solution-form} остаётся справедливой.

\subsubsection{Вычисление коэффициентов ряда}

Для построения ряда \eqref{eq:gs-solution-form} необходимо определить
значения коэффициентов $f_k$.

Первый коэффициент вычисляется элементарно:
\begin{equation}
  f_0 = \intl_0^1 \ch{x} \,dx = \sh{1}
\end{equation}

Для вычисления коэффициентов $f_1, f_2, \dotsc$ найдём значение
интеграла $J_k = \int_0^1{\ch(x) \cos(\pi k x)\,dx}$ (так что $f_k =
\sqrt{2}\cdot J_k$), воспользовавшись интегрированием по частям:
\begin{multline*}
  J_k = \intl_0^1{\ch(x) \cos(\pi k x)\,dx} =
  \underbrace{\intat{\frac{\ch(x)\sin(\pi k x)}{\pi k}}{1}{0}}_0 -
  \intl_0^1\frac{\sh(x) \sin(\pi k x)}{\pi k} \,dx = \\ =
  \underbrace{\vphantom{\intl_0^1}\intat{\frac{\sh(x)\cos(\pi k
        x)}{\pi^2 k^2}}{1}{0}}_{\frac{\sh{1}\cdot (-1)^k}{\pi^2 k^2}}
  - \underbrace{\intl_0^1{\frac{\ch(x) \cos(\pi k x)}{\pi^2
        k^2}\,dx}}_{\frac{J_k}{\pi^2 k^2}}
\end{multline*}
откуда получаем $J_k$:
\begin{gather*}
  J_k = \frac{\sh{1}\cdot (-1)^k}{\pi^2 k^2 + 1}
\end{gather*}

Таким образом, коэффициенты $f_k$ для $k \in \set{N}$ равны
\begin{equation}
  f_k = \frac{\sqrt{2} \cdot \sh{1}\cdot (-1)^k}{\pi^2 k^2 + 1}
\end{equation}

Подставляя теперь известные $f_k, \lambda_k, \psi_k$ и $\lambda = 2$ в
\eqref{eq:gs-solution-form}, получим решение уравнения в виде ряда по
системе функций $\{\psi_i\}$:
\begin{multline}
  \label{eq:gs-solution}
  \phi(x) = f(x) + 2 \sum_{k=0}^\infty \left [
    f_k\cdot\frac{\psi_k(x)}{\lambda_k-2} \right ] = \\
  = f(x) - 2 \sh{1} + 2 \sum_{k=1}^\infty\left [ \frac{\sqrt{2} \cdot
      \sh{1}\cdot (-1)^k}{\pi^2 k^2 + 1} \cdot \frac{\sqrt{2}\cos{\pi
        k x}}{\pi^2 k^2 - 1} \right ] = \\
  = f(x) + \sh{1} \left ( -2 + 4  \sum_{k=1}^\infty \frac{(-1)^k
      \cos{\pi k x}} {\pi^4 k^4 - 1} \right )
\end{multline}

Отметим высокую скорость сходимости ряда \eqref{eq:gs-solution} (как у
ряда $\sum \limits_{k=1}^\infty \frac{(-1)^k}{k^4}$).

\subsection{Реализация}
\label{sec:gs-implementation}

\input{series.mac.tex}

\clearpage
\subsection{Сравнение с аналитическим решением}

На рисунке \ref{fig:series-plot} сплошной линией изображён график
аналитического решения, точками обозначены значения приближённого
решения при соответствующих значениях $x$. Для построения графика была
выбрана частичная сумма ряда из 5 членов.

\input{series-plot-5.tkz.tex}


\clearpage
\section{Численное решение}
\label{sec:numerical}
\subsection{Описание метода}
Воспользуемся квадратурной формулой для вычисления определённого
интеграла $\intl_a^b {K(x, t) \phi(t)\,dt}$ в \eqref{eq:ieqgen}.

Пусть на отрезке интегрирования $[a; b]$ введена равномерная сетка
$\Xi_n = \langle a = \tau_1, \dotsc, \tau_n = b \rangle$ с шагом
$h = \frac{b-a}{n-1}$. Тогда квадратурная формула трапеций принимает
вид (см. \cite{bakhvalov01}):
\begin{equation}
  \label{eq:trapez-rule}
  \intl_a^b{f(x)\,dx} \approx h \left [ \frac{f(\tau_1)+f(\tau_n)}{2} +
    \sum_{j=2}^{n-1} f(\tau_j) \right ]
\end{equation}

С учётом \eqref{eq:trapez-rule} интегральное уравнение
\eqref{eq:ieqgen} в приближённом виде записывается следующим образом:
\begin{equation*}
  \phi(x) - \lambda h \left[ 
    \frac{K(x,\tau_1) \phi(\tau_1) + 
      K(x,\tau_n) \phi(\tau_n)}{2} + 
    \sum_{j=2}^{n-1} K(x, \tau_{j}) \phi(\tau_{j}) 
  \right] = f(x)
\end{equation*}

Записывая его для каждого $x = \tau_i \in \Xi_n$, получим систему линейных
уравнений:
\begin{equation*}
  \phi(\tau_i) - \lambda h \left[ 
    \frac{K(\tau_i,\tau_1) \phi(\tau_1) + 
      K(\tau_i,\tau_n) \phi(\tau_n)}{2} + 
    \sum_{j=2}^{n-1} K(\tau_i, \tau_{j}) \phi(\tau_{j}) 
  \right] = f(\tau_i)
\end{equation*}

В матричном виде эта система имеет вид:
\begin{equation}
  \label{eq:matrix-equation}
  (E_n-\lambda B) \vec{\phi} = \vec{f}
\end{equation}

В системе \eqref{eq:matrix-equation} матрица $E_n$ — единичная размера
$n×n$, а $B$ состоит из элементов $b_j^i$, определяемых как
\begin{equation}
  \label{eq:B-matrix-element}
  b_j^i =
  \begin{cases}
    \frac{1}{2}hK(\tau_i, \tau_j) & j = 1,\, j = n \\
    \phantom{\frac{1}{2}} hK(\tau_i, \tau_j) & j = \overline{2,n-1}
  \end{cases}
\end{equation}

Решение системы \eqref{eq:matrix-equation} представляет собой вектор
$\vec{\phi}$, состоящий из значений функции $\phi(x)$ в точках сетки
$\Xi_n$:
\begin{equation}
  \label{eq:num-solution}
  \vec{\phi} = 
  \begin{pmatrix}
    \phi(\tau_1) \\
    \phi(\tau_2) \\
    \vdots \\
    \phi(\tau_n)
  \end{pmatrix}
\end{equation}

\clearpage
\subsection{Реализация}
\label{sec:numerical-implementation}

\input{numeric.oct.tex}

\subsection{Сравнение с аналитическим решением}
\label{sec:comparison}

На иллюстрации \ref{fig:numeric-plot} сплошной линией обозначен график
аналитического решения, точками обозначены компоненты вектора
$\vec{\phi}$ при соответствующих значениях аргумента.

\input{numeric-plot-10.tkz.tex}

\clearpage
\appendix
\section{Информация о документе}

Данный документ был подготовлен с использованием \LaTeX{}. Для
построения ряда в разделе \ref{sec:gs} применялась система
\program{Maxima}. Программа из раздела
\ref{sec:numerical-implementation} написана в среде
\program{GNU Octave}. Код представлен с использованием
\program{noweb}. Иллюстрации созданы с помощью пакета
\program{pgfplots} и \program{gnuplot}.

Автоматизация процесса сборки обеспечивалась утилитами
\program{GNU Make} и \program{texdepend}.

Представленная работа выполнена в рамках программы пятого семестра
обучения по специальности «Вычислительная математика и математическая
физика» в МГТУ им. Н. Э. Баумана.

Дата компиляции настоящего документа: \today

% \clearpage
\bibliographystyle{gost71s}
\bibliography{paper}

\end{document}
